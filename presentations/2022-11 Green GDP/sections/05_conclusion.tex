\section{Takeaways}

\begin{frame}{Main takeaways}
  \begin{enumerate}
    \item Quality of the water environment improved from 1990-2020.
    \begin{itemize}
      \item If $\Delta\text{GNNI}>\Delta\text{NNI}\Rightarrow$ NNI and GDP underestimated growth.
    \end{itemize}
    \pause
    \item Changes in sociodemographic factors affect the Green NNI.
    \pause
    \item The marginal WTP per year for a water quality of \emph{good} as opposed to \emph{bad} would add up to:
    \begin{itemize}\normalsize
      \item DKK  7 b (2020-prices) for all streams.
      \item DKK  4 b (2020-prices) for all lakes.
      \item DKK  6 b (2020-prices) for all coastal waters.
      \item DKK 13 b (2020-prices) for all groundwater bodies.
    \end{itemize}
  \end{enumerate}
  \note<1>{\textbf{PRELIMINARY RESULTS AND DISCUSSION}\\
    Overall, the quality of ecosystem services has improved since 1990. That is likely to be offset by the costs of GHG emissions and the depletion of exhaustible natural resources\\
    - but if it should turn out that $\Delta\text{GNNI}>\Delta\text{NNP}$,
    \begin{itemize}
      \item[$\Rightarrow$] then it would indicate that GDP growth has not been at the expense of the environment according to the definition of "strong" sustainability.\\\bigskip
    \end{itemize}
    That is, with reservations that we don't fully live up to our international commitment such as the EU Water Framework Directive and the GHG reduction path implied by the Paris Agreement DESPITE outsourcing of our most polluting factories during the period.
  }
  % \note{\textbf{ROBUSTNESS}\\
  %   To construct an unbroken time series, we need to only rely on test methods for ecological and chemical quality that has been applied since the early 90s while applying so-called "heroic assumptions", thus
  %   \begin{itemize}
  %     \item[$\Rightarrow$] \textit{Comprehensive robustness checks are necessary}
  %   \end{itemize}
  %   some of which will have to be "back-of-the-envelope" calculations.
  % }
\end{frame}