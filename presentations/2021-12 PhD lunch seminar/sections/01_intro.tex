\section{Motivation and framework}

\begin{frame}{Why calculate a Green GDP?}
  GDP has become synonymous with welfare despite not capturing:\par
  \begin{enumerate}
    \item The value of the consumption of ecosystem services.
    \item The value of social factors.
  \end{enumerate}
  \vfill
  \note{\textbf{MOTIVATION (1)}\\
    Contrary to Simon Kuznets' warning back in the 1930s where he was in charge of developing the concept of GDP, GDP has largely become synonymous with welfare - which has led to criticism of its shortcomings in not capturing either (1) or (2).\\
    Therefore, there is a widespread search for alternative measures:
    \begin{itemize}
      \item e.g. the EU Commission motivates their "Beyond GDP initiative" as being "about developing indicators that are as clear and appealing as GDP, but more inclusive of environmental and social aspects of progress. Economic indicators such as GDP were never designed to be comprehensive measures of prosperity and well-being."
    \end{itemize}
  }
\end{frame}
\begin{frame}{Why calculate a Green GDP?}
  GDP has become synonymous with welfare despite not capturing:\par
  \begin{enumerate}
    \item The value of the consumption of ecosystem services.
    \item The value of social factors.
  \end{enumerate}
  Our estimation of a \textbf{Danish Green GDP} serves a dual purpose:
  \begin{itemize}
    \item Analyze whether the development from 1990-2020 meets the criterion of "strong" sustainability?
    \item Provide a measure that is directly comparable to the GDP.
  \end{itemize}
  \vfill
  \note{\textbf{MOTIVATION (2)}\\
    As a solution to the first point, we estimate a Danish Green GDP with a dual purpose:
    \begin{itemize}
      \item (...) i.e. a positive net growth in the environmental quality.
      \item using a measure that is directly comparable to the familiar concept of the GDP. The concept of Genuine Saving is less known but still included as a component of the GNNP - which moreover includes the current benefit of the environmental quality.
    \end{itemize}
  }
\end{frame}
\begin{frame}{Why calculate a Green GDP?}
  GDP has become synonymous with welfare despite not capturing:\par
  \begin{enumerate}
    \item The value of the consumption of ecosystem services.
    \item The value of social factors.
  \end{enumerate}
  Our estimation of a \textbf{Danish Green GDP} serves a dual purpose:
  \begin{itemize}
    \item Analyze whether economic development from 1990-2020 meets the criterion of "strong" sustainability?
    \item Provide a measure that is directly comparable to the GDP.
  \end{itemize}
  \begin{tcolorbox}
    \begin{align*}
        \textbf{GNNP} = \text{GDP} &- \text{depreciation of manufactured capital} \\
        &+ \text{net foreign factor income} \\
        &+ \text{\color{green}benefit of the environmental quality} \\
        &+ \text{\color{green}net growth in the environmental quality}
    \end{align*}
  \end{tcolorbox}
  \note{\textbf{RESEARCH FRAMEWORK}\\
    In the literature, the Green NNP is the prefered measure, while one can deduct the Green GDP from it.\\\bigskip
    The \textbf{Green NNP} can be defined as:\\\bigskip
    (...) which is the NNP (before accounting for the environment)\\
    +\textcolor{green}{current marginal benefit of the environmental quality} \\
    +\textcolor{green}{present value of net growth in environmental quality}
    \\\bigskip
    \textbf{[Only if asked - in more general terms:]}\\\bigskip
    GNNP = NNI\\
    +\textcolor{green}{value of consumption of environmental services} \\
    +\textcolor{green}{value of saving in environmental assets} \\
  }
\end{frame}