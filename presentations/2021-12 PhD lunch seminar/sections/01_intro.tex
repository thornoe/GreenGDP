\section{Motivation and framework}

\begin{frame}{Why calculate a Green GDP?}
  \epigraph{"The welfare of a nation can scarcely be inferred from a measurement of national income"}{\textit{Simon Kuznets, 1934}}
  \pause
  GDP has become synonymous with welfare despite not capturing:
  \begin{enumerate}
    \item The value of the consumption of ecosystem services.
    \item The value of social factors.
  \end{enumerate}
  \vfill

  \note<1>{\textbf{MOTIVATION (1)}\\
    While Simon Kuznets' was in charge of developing the concept of GDP in the 1930s, he warned that \textbf{("...")}.\\
  }
  \note<2>{\textbf{MOTIVATION (2)}\\
  Nonetheless, GDP has largely become synonymous with welfare - which has led to criticism of its shortcomings in not capturing either (1) or (2).\\
  Therefore, there is a widespread search for alternative measures
  \begin{itemize}
    \item e.g. the EU Commission has launched a \textbf{Beyond GDP initiative}, motivated as being 
    \textit{"about developing indicators that are as clear and appealing as GDP, but more inclusive of environmental and social aspects of progress. Economic indicators such as GDP were never designed to be comprehensive measures of prosperity and well-being."}
  \end{itemize}
}
\end{frame}



\begin{frame}{Research framework}
  Our estimation of a \textbf{Danish Green GDP} serves a triple purpose:
  \pause
  \begin{enumerate}
    \item Monetary valuation allows summation of ecosystems.
    \pause
    \item Provide a measure that is directly comparable to the GDP.
    \pause
    \item Analyze whether economic development from 1990-2020 meets the criterion of "strong" sustainability?
  \end{enumerate}
  \pause
  \begin{tcolorbox}
    \begin{align*}
        \textbf{GNNP} = \text{GDP} &- \text{depreciation of manufactured capital} \\
        &+ \text{net foreign factor income} \\
        &+ \text{\color{green}benefit of the environmental quality} \\
        &+ \text{\color{green}net growth in the environmental quality}
    \end{align*}
  \end{tcolorbox}
  \vfill

  \note<1>{\textbf{TRIPLE PURPOSE}\\
    As a solution to the first shortcoming of GDP,\\
    we estimate a Danish Green GDP with a triple purpose:
  }
  \note<2>{\textbf{PURPOSE (1)}\\
    \begin{enumerate}
      \item (...) and indicates the relative importance of one ecosystem compared to another.
    \end{enumerate}
  }
  \note<3>{\textbf{PURPOSE (2)}\\
    \begin{enumerate}
      \item 
      \item using a measure that is directly comparable to the familiar concept of the GDP.
      \begin{itemize}
        \item The concept of \textbf{Genuine Saving} is less known but still included as a component of the GNNP - which moreover includes the current benefit of the environmental quality.
      \end{itemize}
    \end{enumerate}
  }
  \note<4>{\textbf{PURPOSE (3)}\\
  \begin{enumerate}
    \item 
    \item 
    \item Neither GDP nor the Green GDP should be interpreted as a measure for welfare, but the Green GDP is the attempt to \textbf{(...)} i.e. whether growth happened at the expense of the overall environment or allowed for a positive net growth in the environmental quality?
  \end{enumerate}
  }
  \note<5>{\textbf{DEFINITION OF THE GNNP}\\
    In the literature, the Green NNP is the preferred measure, while one can deduct the Green GDP from it.\\\bigskip
    The \textbf{Green NNP} can be defined as:\\\bigskip
    (...) is the NNP capturing the annual output of Danish citizens before accounting for the environment\\
    +\textcolor{green}{current marginal benefit of the environmental quality} \\
    +\textcolor{green}{present value of net growth in environmental quality}
    \\\bigskip
    \textbf{[In more general terms - \textit{only if asked}]}\\
    +\textcolor{green}{value of consumption of environmental services} \\
    +\textcolor{green}{value of saving in environmental assets} \\
  }
\end{frame}