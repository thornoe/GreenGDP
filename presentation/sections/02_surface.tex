\section{Overfladevand}

\begin{frame}{Indikatorer for overfladevands økologiske tilstand}
  \textbf{Vandrammedirektivet:}\par
  På baggrund af indikatorer for eutrofiering fastsættes vandområders økologiske tilstand som \textit{høj, god, moderat, ringe} eller \textit{dårlig}.
  \vfill
\end{frame}
\begin{frame}{Indikatorer for overfladevands økologiske tilstand}
  \textbf{Vandrammedirektivet:}\par
  På baggrund af indikatorer for eutrofiering fastsættes vandområders økologiske tilstand som \textit{høj, god, moderat, ringe} eller \textit{dårlig}.\par
  \vspace{0.5\baselineskip}
  Vi avender indikatorer, der har været målt på standardiseret vis siden henholdsvis 1989 og 1992:
  \begin{itemize}
    \item \textbf{Vandløb:} Sammensætningen af \textit{smådyr} (1992).
    \item \textbf{Søer:} Koncentrationen af stoffet \textit{klorofyl a} som estimat for biomassen af planteplankton (1989).
    \item \textbf{Fjorde og kystvande:} Dybdeudbredelse af \textit{ålegræs} (1989).
    \begin{itemize}
      \item Vestkysten og Vadehavet: Koncentrationen af \textit{klorofyl} (1989).
    \end{itemize}
  \end{itemize}
\end{frame}

